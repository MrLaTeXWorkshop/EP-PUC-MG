 %------Questao1--------------------------------------------------------------------------------%

 \section*{Respostas}

 \begin{question}
       
    \begin{enumerate}[label={\textbf{\alph*)}}]

        \item 
        
        Amostragem refere-se ao processo ou técnica de escolha e seleção de membros de uma população, 
        ou de um universo estatístico que possam constituir uma amostra.
        \item 
        
        Amostragem aleatória simples é um tipo de amostragem probabilística, onde todos os elementos 
        da população têm a mesma probabilidade de estarem presentes em uma amostra, já que 
        o método de seleção é um sorteio.
        \item 
        
        \begin{itemize}
            \item Aplicar um questionário de satisfação sobre os serviços prestados por uma agência bancária
            em 10 clientes de um banco de dados de 100 pessoas. Verifica-se que das 100 pessoas 30\% sãomulheres
            e  70\%  são  homens.  Delimita-se  que  dos  10  clientes  aserem entrevistados 3 devem ser mulheres e 7 homens.

            \item Será realizada uma pesquisa com 200 estudantes de uma população de 10 mil. Suponhamos que o
            grupo de alunos dessa instituição seja composto de 30\% de calouros, 30\% de estudantes do segundo ano, 20\% do terceiro e 20\% do último.
        \end{itemize}
        \item 
        
        \begin{itemize}
            \item Para obter uma amostra de famílias: selecionar primeiro uma amostra de cidades; selecionar bairros das cidades sorteadas; selecionar
            quarteirões dos bairros sorteados; selecionar domicílios dos quarteirões sorteados.

            \item Vamos supor que você esteja trabalhando na campanha eleitoral de um candidato a governador do Distrito Federal. Você pretende enviar
            uma pesquisa de opinião para saber a intenção de voto dos eleitores. Até agora, você possui as seguintes informações: População a ser analisada: residentes do Distrito Federal.
            Uma lista das regiões administrativas do Distrito Federal.
        \end{itemize}
        \item 
        
        \begin{itemize}
            \item Se a população do estudo contém 2000 estudantes do ensino fundamental e o pesquisador quer uma amostra de 100 estudantes. Os estudantes
            poderiam ser colocados em uma lista e cada 20º estudante seria selecionado para inclusão na amostra. A fim de evitar o viés humano neste método,
            o pesquisador deve selecionar o primeiro elemento aleatoriamente. 

            \item Suponhamos um marco amostral de 5.000 indivíduos e desejamos obter uma amostra com 100 deles. Em primeiro lugar, dividimos o marco amostral
            em 100 fragmentos de 50 indivíduos. Selecionamos um número aleatório entre 1 e 50 para extrair o primeiro indivíduo de forma aleatória: por exemplo
            o número 24.
        \end{itemize}

    \end{enumerate}
\end{question}