%------Questao3--------------------------------------------------------------------------------%

\begin{question}

    \begin{formula1}
        {54}{908}{332}{70836}{3724}
    \end{formula1}
    
    \begin{enumerate}[label={\textbf{\alph*)}}]

        \item

        \begin{formula6}
            {12}{332}{54}{3984}{2916}{1068}
        \end{formula6}

        $\\$

        \begin{formula7}
            {12}{70836}{908}{850032}{824464}{25568}
        \end{formula7}

        $\\$

        \begin{formula8}
            {12}{3724}{49032}{44688}{-4344}
        \end{formula8}

        $\\$

        \begin{formula5}
            {-4344}{1068}{25568}{5225.57}{-0.8313}
        \end{formula5}

        \textbf{Interpretação do resultado:} Há uma forte correlação negativa.

        \item  

        \begin{formula9}
            {-4344}{1068}{-4.07}{75.66}{4.5}{93.97}
        \end{formula9}

        \textbf{Interpretação do resultado:}

        $\hat{\beta}0$ : Para os alunos que não gastam tempo nenhum na frente da televisão, é esperado que 
        uma nota média de 93.97 pontos na prova de segunda. \\
        $\hat{\beta}1$ : Para cada hora adicional na frente da televisão, é esperado um decréscimo na 
        média em 4.07 pontos na nota.

        \item  

        $R^2$ = $(0.8313)^2$ = 0.691 ou 69.10\%

        \textbf{Interpretação do resultado:} 

        A variabilidade da pontuação do teste de 69.10\%, pode ser explicado pelas horas na
        frente da televisão. Os restantes 30.90\% não possuem explicação clara, logo pode ser 
        resultado de um possível erro, ou outra variável que não foi incluída na questão.

        \item 

        \begin{formula10}
            {70836}{93.97}{908}{4.07}{3724}{10}{667.92}{8.17}
        \end{formula10}

        $\\$

        \begin{formulaA}
            {93.97}{4.07}{9}{57.34}
        \end{formulaA}

        \begin{formula2}
            {95}{57.34}{2.2281}{8.17}{12}{9}{4.5}{1068}
        \end{formula2}

        \begin{formula4}
            {95}{57.34}{20.84}{36.5}{78.18}
        \end{formula4}

        \item 

        \begin{formula3}
            {95}{57.34}{2.2281}{8.17}{12}{9}{4.5}{1068}
        \end{formula3}

        \begin{formula4}
            {95}{57.34}{10.15}{47.19}{67.49}
        \end{formula4}
    \end{enumerate}
\end{question}

